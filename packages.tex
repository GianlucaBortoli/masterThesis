\usepackage{epsfig}
\usepackage{plain}
\usepackage{setspace}
\usepackage[paperheight=29.7cm,paperwidth=21cm,outer=1.5cm,inner=2.5cm,top=2cm,bottom=2cm]{geometry} % per definizione layout
\usepackage{titlesec} % per formato custom dei titoli dei capitoli
\usepackage[utf8x]{inputenc} % per Linux (richiede il pacchetto unicode);
\usepackage{url}
% my packages
\usepackage{graphicx}
\graphicspath{{imgs/}} % Specifies the directory where pictures are stored
\usepackage{amsmath}
\usepackage{amsfonts}
\usepackage{amssymb}
\usepackage{mathtools}
\DeclarePairedDelimiter{\ceil}{\lceil}{\rceil}

\usepackage{listings} % Source code support
\usepackage{color}

% syntax highlighting start
\definecolor{lightgray}{rgb}{.9,.9,.9}
\definecolor{darkgray}{rgb}{.4,.4,.4}
\definecolor{purple}{rgb}{0.65, 0.12, 0.82}

\lstdefinelanguage{JavaScript}{
    keywords={typeof, new, true, false, catch, function, return, null, catch, switch, var, let, if, in, while, do, else, case, break},
    keywordstyle=\color{blue}\bfseries,
    ndkeywords={class, export, boolean, throw, implements, import, this},
    ndkeywordstyle=\color{darkgray}\bfseries,
    identifierstyle=\color{black},
    sensitive=false,
    comment=[l]{//},
    morecomment=[s]{/*}{*/},
    commentstyle=\color{purple}\ttfamily,
    stringstyle=\color{red}\ttfamily,
    morestring=[b]',
    morestring=[b]",
    morestring=[b]`
}
\lstset{
    language=JavaScript,
    backgroundcolor=\color{lightgray},
    extendedchars=true,
    basicstyle=\footnotesize\ttfamily,
    showstringspaces=false,
    showspaces=false,
    numbers=left,
    numberstyle=\footnotesize,
    numbersep=9pt,
    tabsize=2,
    breaklines=true,
    showtabs=false,
    captionpos=b
}
% syntax highlighting end