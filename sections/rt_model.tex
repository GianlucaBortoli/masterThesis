%!TEX root=../thesis.tex
\chapter{A Real-Time model for WebGL} \label{cha:rt_model}

WebGL is a relatively new technology that enables hardware-accelerated web
contents. Moreover, in the last few years Virtual Reality (VR) has became
another very popular research topic. The Mozilla Foundation is actively working
for porting VR technologies to the Web my means of WebVR~\cite{mozvr}. WebVR is
``an open specification that makes it possible to experience VR'' inside a web
browser, whose main goal is to make VR experiences easier regardless of the
device used. This specification~\cite{webvrspecs} is made available to developers
through different frameworks such as Mozilla's A-Frame~\cite{aframe}, which allows
to build virtual reality scenes inside the browser using only HTML and JavaScript
code. This may seem a very different approach with respect to the one of this
thesis, but what makes A-Frame similar to Cesium is how they exploit WebGL to
interact with the GPU to achieve hardware acceleration. In this sense it is
reasonable to compare VR and 3D map web technologies since they both use WebGL
as its core part. Furthermore, virtual reality applications are even more
performance-demanding, since large latencies jeopardize the entire user
experience.


\section{Assumptions}
First of all, all the experiments presented in this thesis are run on a Dell
Inspiron 15R 5521 notebook\footnote{\url{http://www.dell.com/us/dfh/p/inspiron-15r-5521/pd}.},
whose hardware specifications are shown in Table \ref{tab:notebook_specs}.
Since graphics performances highly depends on the GPU available to the system,
the results presented in Chapter \ref{cha:experiments} are likely to be overtaken
if better graphics hardware (eg. a dedicated GPU) is installed.
\begin{table}[!htb]
    \centering
    \caption{The test machine hardware specifications.}
    \label{tab:notebook_specs}
    \begin{tabular}{|l|l|l|l|}
    \hline
    \multicolumn{1}{|l|}{\textbf{CPU}} & \multicolumn{1}{l|}{\textbf{GPU (integrated)}} & \multicolumn{1}{l|}{\textbf{RAM}} & \multicolumn{1}{l|}{\textbf{Hard drive}} \\ \hline
    Intel i7 3537U & Intel® HD Graphics 4000 & 8GB DDR3 & Samsung 850 Evo 250GB \\ \hline
    \end{tabular}
\end{table}

Another important aspect is how tests are conceived. As it is always the case in
real-time analysis, the worst possible scenario is the interesting one for
computing the WCET of a task. To achieve this situation using the test application,
Google Chrome is always run from scratch with only one active tab. Moreover, the browser
is forced not to cache any part of the application code but, at the same time, it
is essential to get rid of the map tils' retrival time. Usually map tiles are
downloaded via HTTP using the REST APIs a terrain server exposes to the clients.
Even the time the navigator takes to download a small portion of the terrain
can be orders of magniture bigger than all the rest of the response time.
Hence, this particular task is obtained allowing the browser to save locally the
tiles downloaded during a ``warm-up'' phase. This allows to significantly lower
the map retrival time during the ``real'' experiment that is later discarded from
the trace.

Finally, picking timestamps during the execution of a piece of software can bias
the correctness of the timings themselves. This is another point that has to be
taken into account during any type of analysis involving time measurements like
the one presented in this work. In simple terms, measuring time takes time.


\section{Tools used}
This section presents the tools used to trace function calls and to pick the needed
timings in the experiments (see Chapter \ref{cha:experiments}).

\subsection{Web Tracing Framework}
The Web Tracing Framwork~\cite{wtf} (WTF) is composed of a set of tools for
instrumenting, analyzing and visualizing web application execution traces.
It is an extensible framework able to capture and replay HTML5 canvas elements
with its WebGL content. In this way it is possible to implement a plugin for WTF
for the specific purpost of the application under test. Furthermore, a wide set
of methods and events can be tracked out-of-the-box for preliminar investigations.

WTF is a very good toolset for an high-level analysis of web applications,
providing easy code instrumentation to track only what is truly interesting.
The usual work-flow is the following. First, the developer either selects the events
to be traced from a predefined set or he/she implements its own specific code
instrumentation with the available APIs. After that, the application's source code
has to include three function calls to prepare, start and stop tracing.
An example of application source code is the following:
\begin{lstlisting}[caption=Usage example of the Web Tracing Framework., language=JavaScript,
    label=code:wtf_example]
    let wtfOptions = { ... }; // customize WTF behaviour
    wtf.trace.prepare(wtfOptions);
    /* application initialization */
    wtf.trace.start(); // start profiling
    /* code to be traced goes here */
    wtf.trace.snapshot("file://trace"); // save trace to file
    wtf.trace.stop(); // stop profiling
\end{lstlisting}

After taking a snapshot of the current trace (line 6) and WTF is stopped (line 7,
code snippet \ref{code:wtf_example}), the trace is exported in a
compressed format that can be manipulated using some CLI
scripts\footnote{\url{https://github.com/google/tracing-framework}.} to export
data in CSV format or it can be visualized inside a web
page\footnote{\url{http://google.github.io/tracing-framework/bin/app/maindisplay.html}.}.

Another highlight is its extendibility. In this way, the developer can either
profile the application using events emitted by the browser by default or the
source code can be instrumented to track very specific parts. Basically, this
allows to measure nearly everything that is accessible from JavaScript, even
though it may not be enough in some situation.



\subsection{Chrome's trace event profiler}



