%!TEX root=../thesis.tex
\chapter{Introduction} \label{cha:intro}

In the last few years companies are putting a huge effort in developing a wide variety of 
web frameworks to create Graphical User Interfaces (GUIs). Let us think about the success of Facebook's
React~\cite{fbreact} and Google's AngularJS~\cite{angularjs}, which are two of the most
used and famous javascript libraries to build user interfaces across the web.
These technologies are very well known to websites developers which use them to
build their products more easily.\\
Talking about the enterprise environment, most of the desktop pieces of software have its own GUI
written in a native programming language that is often the same throughout the entire application.
One of the most common examples is Java~\cite{gosling1995java}, which makes use of the JavaFX~\cite{javafx}
platform or the Swing~\cite{javaswing} toolkit to build the graphics.\\
What may sound surprising at a first glance is the trend of building user interfaces taking
advantage of web technologies. They allow to design and prototype UIs very quickly and, in
most of the cases, they give also better-looking results.
A very good example of the usage of such technologies is NASA's Open MCT~\cite{openmct}.
OpenMCT is a next-generation mission control software developed both for desktop and mobile
usage. It has been used for mission planning and operations in the Resouce 
Prospector mission~\cite{andrews2014introducing} at NASA's Research Center as well as for
data analysis of spacecraft missions. This software is clearly the evidence that web
technologies have a huge potential that goes beyond the mere website development and that they 
can be applied to deliver highly critical services.
Another kind of web application that is becoming very popular is 2D and 3D map navigators.
Let us think to the technologies lying in the core of very successfull products like
Google Maps, Microsoft's Bing Maps and OpenStreetMap.\\
Given that running UIs inside a browser window is gaining popularity, both inside the
developer communities and the companies, one of the main aspects that has to be analyzed
in detail is the performances that can be delivered and guaranteed. This is probably one of
the hardest tasks a web developer has to face building an application. Finding bugs
or bottlenecks inside the code can be a draining and tricky activity due to the huge number
of components and layers involved to keep the application up and running. Furthermore,
being able to prove that a web application respects some kind of temporal constraint
is even harder or, in some case, impossible. Hence, the main aim of this thesis is
to build an analysis framework able to address and to study in detail the
abovementioned problems.

The test application developed and analized in this work is a 3D-capable navigator.
This application is one of the main requirements for the ACANTO~\cite{acanto}
project\footnote{This project has received funding from the European Union’s Horizon
2020 research and innovation programme - Societal Challenge 1 (DG CONNECT/H) under 
grant agreement No 643644.}, more specifically as part of the FriWalk robotic walker.
The work field of ACANTO is to study and develop an effective strategy to fight the
physical and cognitive decline of older adults in the face of ever shrinking financial
resources for health care and social services. This is achieved by means of FriWalk,
a robotic walking assistant that supports the user in its dayly activities requiring
physical exercise. Furthermore, the system has to reccomend activities selected 
by a senior user (eg. a doctor) to the user.
