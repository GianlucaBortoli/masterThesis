%!TEX root=../thesis.tex
\chapter*{Abstract} \label{abstract} % senza numerazione
\addcontentsline{toc}{chapter}{Abstract} % da aggiungere comunque all'indice

Recent surveys report that web technologies are very popular not only among
website developers but also to build graphical user interfaces for corporate software.
This trend sees WebGL as one of its main actors, thanks to its ability to create
3D-capable applications exploiting hardware acceleration. For this reason, WebGL
is becoming the ``de facto'' standard API for 3D graphics in web applications and
it is currently supported by all the major browsers.
%
Given that this approach to 3D graphics is widely employed, a measure for its
real performance is needed. To achieve this, precise calculations of the response
and the computation times of the different layers belonging to the rendering
pipeline have to be collected.
In addition, another goal of this thesis is to investigate if WebGL is able to
respect real-time temporal constraints. As far as we know, at the time of writing
there is still no science literature applying real-time models to web technologies.
%
The application under test is a desktop 3D map navigator specifically built for
a robotic walking assistant. Moreover, the performance measurement phase is
accomplished using two ad-hoc tools: the Web Tracing Framework and Chrome's
Event Profiling Tool. The first is used for the very first investigations of the
response time and to get insights of what is happening from an high level point
of view. On the other hand, the profiler embedded inside the Chrome browser is
able to capture exact computation times of the rendering primitives with
microsecond granularity. This advanced profiler allows to overtake the limits
of the other framework, namely the inability to retrieve computation times
instead of just response times and the level of detail that was clearly not
enough for applying any real-time mathematial model.

% TODO:
% * results and their significance
% * future research
