%!TEX root=../thesis.tex
\chapter*{Abstract} \label{abstract} % no numbering
\addcontentsline{toc}{chapter}{Abstract} % add to index anyway

The purpose of this work is to verify if it is feasible to embed a 3D navigator
inside the FriWalk's system architecture. This application is developed using
WebGL, the ``de facto'' standard for 3D hardware-accelerated graphics in
web applications. Therefore it is essential to understand if such technology
is appropriate for building user interfaces in the real-time field, where
providing temporal guarantees on the response time is a mandatory constraint.
As far as we know, at the time of writing there is still no science literature
applying real-time models to web technologies.
%
This work analyzes in detail all the layers which are part of the WebGL
rendering pipeline and it provides exact measurements for the computation time
of every single rendering task.
This performance assessment phase is accomplished using two ad-hoc tools: the
Web Tracing Framework and Chrome's Event Profiling Tool.
The first is used for the very first investigations of the response time and to
get insights of what is happening from an high level point of view.
On the other hand, the profiler included inside the Chrome browser is able to
capture exact computation times of the rendering primitives with microsecond
granularity.
%
Such precision is essential for applying real-time mathematical models which
define probabilistic deadlines and that are able to predict the trend of the
entire system.
%
The experimental results show that WebGL is able to achieve quite good results.
This is remarkable especially if we consider that the test application is
developed using a very high-level programming language and that it runs inside
a browser, which is rather singular for real-time programs. 
