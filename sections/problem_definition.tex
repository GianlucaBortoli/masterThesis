%!TEX root=../thesis.tex
\chapter{Problem definition} \label{cha:problem_definition}

When developers write applications inside a web browser, they want them to run
as smooth as possible. Making sure they not suffer of performance problems really
makes the difference between a native-like feeling and a cluncky experience.
If these performance issues are not hard enough to achieve, real-time software
inside a web page is even more challenging. Hence, being able to state whether
web technologies are good enough to fit in the soft real-time field is still a
huge and open question. Nobody has applied real-time theory concepts to web
technologies before in the computer science literature. This work aims to find
an answer to this question.
%As far as we know there is no science literature which applies real-time theory to 
%web technologies. 
%(nobody meglio non scriverlo mai magari poi si trova un paper in qualche conferenza indiana)

From an high-level point of view it is possible to say that there are a lot of
%Cambierei a lot con different (a lot of non e' molto preciso e presuppone un confronto)
layers responsible for displaying something from a web page to the screen.
During this work there are two main issues that have to be solved in order to
%cambierei solved con addressed
give an answer to the question:
%qua non la frase non e' molto chiara, secondo me � meglio se riprendi la domanda. 
\begin{itemize}
    \item discovering which are all the components involved in the graphics
        pipeline and stating which portion of the system is responsible for a
        certain job.
    \item measuring time to give a concrete representation of the latencies
        introduced by every layer of abstraction.
\end{itemize}

Once these two points are solved, it is possible to describe the timings by means
of real-time mathematical models and saying if an application is ``fast enough''
for the specific purpose of the application.
%Questa frase � sbagliata nel contesto della tua tesi. Nel real-time non si parla mai di 
%veloce abbastanza, la frase dovrebbe essere qualcosa se l'applicazione e' in grado di 
%soddisfare i requisiti temporali dettati dallo specifico utilizzo
%if an application is able to fullfill the time constraints coming from its specific context. 

This thesis is relevant for its novel approach to soft real-time, studying
%non � proprio vero, is relevant for real-time theory for its novel field of application, 
if web technologies can fit into a field which is mainly composed of computer
programs written in low-level programming languages like C~\cite{kernighan2006c}
and C++~\cite{stroustrup1995c++} due to the high-demanding time constraints.
As previously mentioned in Chapter \ref{cha:intro}, the feasibility study this
been accomplished using the 3D navigator application.
