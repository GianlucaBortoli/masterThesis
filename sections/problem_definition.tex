%!TEX root=../thesis.tex
\chapter{Problem definition} \label{cha:problem_definition}

When developers write applications inside a web browser, they want them to run
as smooth as possible. Making sure they not suffer of performance problems really
makes the difference between a native-like feeling and a clunky experience.
If these performance issues are not hard enough to achieve, real-time software
inside a web page is even more challenging. Hence, being able to state whether
web technologies are good enough to fit in the soft real-time field is still a
huge and open question. As far as we know there is no science literature which
applies real-time theory to web technologies. This work aims to find an answer
to this question.

From an high-level point of view it is possible to say that there are different
layers responsible for displaying something from a web page to the screen.
There are two main issues that have to be addressed in order to discover if web
applications fits the real-time theory:
\begin{itemize}
    \item discovering which are all the components involved in the graphics
        pipeline and stating which portion of the system is responsible for a
        certain job.
    \item measuring time to give a concrete representation of the latencies
        introduced by every layer of abstraction.
\end{itemize}

Once these two points are solved, it is possible to describe the timings by means
of real-time mathematical models and saying if an application is able to fulfill
the time constraints applicable the specific context.

This thesis is relevant in real-time theory for its novel field of application, studying
if web technologies can fit into a field which is mainly composed of computer
programs written in low-level programming languages like C~\cite{kernighan2006c}
and C++~\cite{stroustrup1995c++} due to the high-demanding time constraints.
As previously mentioned in Chapter \ref{cha:intro}, the feasibility study this
been accomplished using the 3D navigator application.
