%!TEX root=../thesis.tex
\chapter{Conclusion and future work} \label{cha:conclusion}

% questo approccio può essere esteso per studiare come si comporta una qualsiasi
% applicatione che gira dentro un browser e NON solamente quella di test

Adopting the Markov Computation Time Model to study the behaviour of a web
application which heavily use WebGL is feasible.
The outcomes presented in Section \ref{sec:mctm_results} show that it is
reasonable to split the original computation time trace into two states, since
the output model is always conservative.
However, the values belonging to each modes are not highly correlated but they
are not independent as they should be in principle.
This may be a problem in the logic of the HMM decoder algorithm. In the specific
case of this work, the classification phase splits the original trace into two
states (Figure \ref{img:mctm_2_states}) whose CDFs are clearly ditinct. On the
other hand, when the system is composed of three modes, two out of three CDFs
are completely overlapped. Hence, two is the number of states that best fit the
MCTM

As future work, it is possible to switch from the ``raw'' computation times taken
into consideration in this thesis to ``per-frame'' values. This can be done
summing the computation times belonging to the same frame and it allows to study
the WebGL rendering performance from a slightly different standpoint.
Furthermore, the work of this thesis is not bound only to the specific
application under test and it can be easily extended to study the performances
of any piece of software running inside a web browser. This flexibility allows
the team working on the ACANTO project to run the entire analysis on the
specific hardware available on the FriWalk. In addition to this, another
challenging improvement is to modify Cesium's rendering pipeline to introduce
the concept of priority queues for the graphics commands. In this way it would
be possible to render the ``most important'' parts of the scene first (eg. near
the user's position) and to delay the ``less important'' ones.
