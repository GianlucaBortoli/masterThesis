%%%%%%%%%%%%%%%%%%%%%%%%%%%%%%%%%%%%%%%%%%%%%%%
%
% Template per Elaborato di Laurea
% DISI - Dipartimento di Ingegneria e Scienza dell’Informazione
%
% update 2015-09-10
%
% Per la generazione corretta del
% pdflatex nome_file.tex
% bibtex nome_file.aux
% pdflatex nome_file.tex
% pdflatex nome_file.tex
%
%%%%%%%%%%%%%%%%%%%%%%%%%%%%%%%%%%%%%%%%%%%%%%%

% formato FRONTE RETRO
\documentclass[epsfig,a4paper,11pt,titlepage,twoside,openany]{book}
% import packages
\usepackage{epsfig}
\usepackage{plain}
\usepackage{setspace}
\usepackage[paperheight=29.7cm,paperwidth=21cm,outer=1.5cm,inner=2.5cm,top=2cm,bottom=2cm]{geometry} % per definizione layout
\usepackage{titlesec} % per formato custom dei titoli dei capitoli
\usepackage[utf8x]{inputenc} % per Linux (richiede il pacchetto unicode);
\usepackage{url}

\singlespacing

\begin{document}
    % Frontispiece
    % nessuna numerazione
    \pagenumbering{gobble}
    \pagestyle{plain}
\thispagestyle{empty}

\begin{center}
    \begin{figure}[h!]
        \centerline{\psfig{file=imgs/unitn_logo.png,width=0.6\textwidth}}
    \end{figure}

    \vspace{2 cm}
    \LARGE{Department of Information Engineering and Computer Science\\}

    \vspace{1 cm}
    \Large{Master's degree in\\
        Computer Science
    }

    \vspace{2 cm}
    \Large\textsc{Final dissertation\\}

    \vspace{1 cm}
    \Huge\textsc{Real-time analysis of WebGL rendering}

    \vspace{2 cm}
    \begin{tabular*}{\textwidth}{ c @{\extracolsep{\fill}} c }
        \Large{Advisor} & \Large{Student}\\
        \Large{Prof. Luigi Palopoli}& \Large{Gianluca Bortoli}\\
    \end{tabular*}

    \vspace{2 cm}
    \Large{Accademic year 2016/2017}
\end{center}


    \clearpage

    % Acknowledgements
    %!TEX root=../thesis.tex
\thispagestyle{empty}

\begin{center}
  {\bf \Huge Acknowledgements}
\end{center}

\vspace{4cm}


\emph{
  ...thanks to...
}

    \clearpage
    \pagestyle{plain} % nessuna intestazione e pie pagina con numero al centro

    % inizio numerazione pagine in numeri arabi
    \mainmatter
    % indice
    \tableofcontents
    \clearpage

    % gruppo per definizone di successione capitoli senza interruzione di pagina
    \begingroup
        % nessuna interruzione di pagina tra capitoli
        % ridefinizione dei comandi di clear page
        \renewcommand{\cleardoublepage}{}
        \renewcommand{\clearpage}{}
        % redefinizione del formato del titolo del capitolo
        % da formato
        %   Capitolo X
        %   Titolo capitolo
        % a formato
        %   X   Titolo capitolo

        \titleformat{\chapter}
            {\normalfont\Huge\bfseries}{\thechapter}{1em}{}

        \titlespacing*{\chapter}{0pt}{0.59in}{0.02in}
        \titlespacing*{\section}{0pt}{0.20in}{0.02in}
        \titlespacing*{\subsection}{0pt}{0.10in}{0.02in}

        % sommario
        %!TEX root=../thesis.tex
\chapter*{Abstract} \label{abstract} % no numbering
\addcontentsline{toc}{chapter}{Abstract} % add to index anyway

The purpose of this work is to verify if it is feasible to embed a 3D navigator
inside the FriWalk's system architecture. This application is developed using
WebGL, the ``de facto'' standard for 3D hardware-accelerated graphics in
web applications. Therefore it is essential to understand if such technology
is appropriate for building user interfaces in the real-time field, where
providing temporal guarantees on the response time is a mandatory constraint.
As far as we know, at the time of writing there is still no science literature
applying real-time models to web technologies.
%
This work analyzes in detail all the layers which are part of the WebGL
rendering pipeline and it provides exact measurements for the computation time
of every single rendering task.
This performance assessment phase is accomplished using two ad-hoc tools: the
Web Tracing Framework and Chrome's Event Profiling Tool.
The first is used for the very first investigations of the response time and to
get insights of what is happening from an high level point of view.
On the other hand, the profiler included inside the Google Chrome browser is
able to capture exact computation times of the rendering primitives with microsecond
granularity. Hence, this approach starts from the ``macro'' situation and it
proceeds focusing on very detiled rendering primitives.
%
Capturing very precise timing values is essential for applying real-time
mathematical models which are able to define probabilistic deadlines and to predict
the trend of the entire system.
%
The experimental results show that WebGL is able to achieve good performances
and that its rendering pipeline can be modeled using a Markov Computation Time
Model.
This goal is remarkable especially if we consider that the test application is
developed using a very high-level programming language and that it runs inside
a browser, which is rather singular for real-time programs.

        % lista dei capitoli
        \input{sections/chapter1}
        \input{sections/chapter2}
        \input{sections/chapter3}
    \endgroup


    % bibliografia in formato bibtex
    % aggiunta del capitolo nell'indice
    \addcontentsline{toc}{chapter}{Bibliography}
    % stile con ordinamento alfabetico in funzione degli autori
    \bibliographystyle{plain}
    \bibliography{biblio}
%%%%%%%%%%%%%%%%%%%%%%%%%%%%%%%%%%%%%%%%%%%%%%%%%%%%%%%%%%%%%%%%%%%%%%%%%%
%%%%%%%%%%%%%%%%%%%%%%%%%%%%%%%%%%%%%%%%%%%%%%%%%%%%%%%%%%%%%%%%%%%%%%%%%%
%% Nota
%%%%%%%%%%%%%%%%%%%%%%%%%%%%%%%%%%%%%%%%%%%%%%%%%%%%%%%%%%%%%%%%%%%%%%%%%%
%% Nella bibliografia devono essere riportati tutte le fonti consultate
%% per lo svolgimento della tesi. La bibliografia deve essere redatta
%% in ordine alfabetico sul cognome del primo autore.
%%
%% La forma della citazione bibliografica va inserita secondo la fonte utilizzata:
%%
%% LIBRI
%% Cognome e iniziale del nome autore/autori, la data di edizione, titolo, casa editrice, eventuale numero dell’edizione.
%%
%% ARTICOLI DI RIVISTA
%% Cognome e iniziale del nome autore/autori, titolo articolo, titolo rivista, volume, numero, numero di pagine.
%%
%% ARTICOLI DI CONFERENZA
%% Cognome e iniziale del nome autore/autori (anno), titolo articolo, titolo conferenza, luogo della conferenza (città e paese), date della conferenza, numero di pagine.
%%
%% SITOGRAFIA
%% La sitografia contiene un elenco di indirizzi Web consultati e disposti in ordine alfabetico.
%% E’ necessario:
%%   Copiare la URL (l’indirizzo web) specifica della pagina consultata
%%   Se disponibile, indicare il cognome e nome dell’autore, il titolo ed eventuale sottotitolo del testo
%%   Se disponibile, inserire la data di ultima consultazione della risorsa (gg/mm/aaaa).
%%%%%%%%%%%%%%%%%%%%%%%%%%%%%%%%%%%%%%%%%%%%%%%%%%%%%%%%%%%%%%%%%%%%%%%%%%
%%%%%%%%%%%%%%%%%%%%%%%%%%%%%%%%%%%%%%%%%%%%%%%%%%%%%%%%%%%%%%%%%%%%%%%%%%

    \titleformat{\chapter}
        {\normalfont\Huge\bfseries}{Appendix \thechapter}{1em}{}
    % sezione Allegati - opzionale
    \appendix
    \input{sections/appendix}

\end{document}
