%%%%%%%%%%%%%%%%%%%%%%%%%%%%%%%%%%%%%%%%%%%%%%%
%
% Template per Elaborato di Laurea
% DISI - Dipartimento di Ingegneria e Scienza dell’Informazione
%
% update 2015-09-10
%
% Per la generazione corretta del
% pdflatex nome_file.tex
% bibtex nome_file.aux
% pdflatex nome_file.tex
% pdflatex nome_file.tex
%
%%%%%%%%%%%%%%%%%%%%%%%%%%%%%%%%%%%%%%%%%%%%%%%

% front-back format
\documentclass[epsfig,a4paper,11pt,titlepage,twoside,openany]{book}
% import packages
\usepackage{epsfig}
\usepackage{plain}
\usepackage{setspace}
\usepackage[paperheight=29.7cm,paperwidth=21cm,outer=1.5cm,inner=2.5cm,top=2cm,bottom=2cm]{geometry} % per definizione layout
\usepackage{titlesec} % per formato custom dei titoli dei capitoli
\usepackage[utf8x]{inputenc} % per Linux (richiede il pacchetto unicode);
\usepackage{url}

\singlespacing

% solves problem to ensure vertical mode for clean[double]page
% https://goo.gl/TxPyga
\usepackage{etoolbox}
\makeatletter
\patchcmd{\chapter}{\if@openright\cleardoublepage\else\clearpage\fi}{\par}{}{}
\makeatother


\begin{document}
    % no page numbering
    \pagenumbering{gobble}
    % frontispiece 
    \pagestyle{plain}
\thispagestyle{empty}

\begin{center}
    \begin{figure}[h!]
        \centerline{\psfig{file=imgs/unitn_logo.png,width=0.6\textwidth}}
    \end{figure}

    \vspace{2 cm}
    \LARGE{Department of Information Engineering and Computer Science\\}

    \vspace{1 cm}
    \Large{Master's degree in\\
        Computer Science
    }

    \vspace{2 cm}
    \Large\textsc{Final dissertation\\}

    \vspace{1 cm}
    \Huge\textsc{Real-time analysis of WebGL rendering}

    \vspace{2 cm}
    \begin{tabular*}{\textwidth}{ c @{\extracolsep{\fill}} c }
        \Large{Advisor} & \Large{Student}\\
        \Large{Prof. Luigi Palopoli}& \Large{Gianluca Bortoli}\\
    \end{tabular*}

    \vspace{2 cm}
    \Large{Accademic year 2016/2017}
\end{center}


    \clearpage

    % acknowledgements
    %!TEX root=../thesis.tex
\thispagestyle{empty}

\begin{center}
  {\bf \Huge Acknowledgements}
\end{center}

\vspace{4cm}


\emph{
  ...thanks to...
}

    \clearpage
    \pagestyle{plain} % nessuna intestazione e pie pagina con numero al centro

    % inizio numerazione pagine in numeri arabi
    \mainmatter
    % indice
    \tableofcontents
    \clearpage

    % gruppo per definizone di successione capitoli senza interruzione di pagina
    \begingroup
        % nessuna interruzione di pagina tra capitoli
        % ridefinizione dei comandi di clear page
        \renewcommand{\cleardoublepage}{}
        \renewcommand{\clearpage}{}

        % redefinizione del formato del titolo del capitolo
        % da formato
        %   Capitolo X
        %   Titolo capitolo
        % a formato
        %   X   Titolo capitolo
        \titleformat
            {\chapter} % command
            {\normalfont\Huge\bfseries} % format
            {\thechapter} % label
            {1em} % sep
            {} % before-code
        \titlespacing*{\chapter}{0pt}{0.59in}{0.02in}
        \titlespacing*{\section}{0pt}{0.20in}{0.02in}
        \titlespacing*{\subsection}{0pt}{0.10in}{0.02in}

        % abstract
        %!TEX root=../thesis.tex
\chapter*{Abstract} \label{abstract} % no numbering
\addcontentsline{toc}{chapter}{Abstract} % add to index anyway

The purpose of this work is to verify if it is feasible to embed a 3D navigator
inside the FriWalk's system architecture. This application is developed using
WebGL, the ``de facto'' standard for 3D hardware-accelerated graphics in
web applications. Therefore it is essential to understand if such technology
is appropriate for building user interfaces in the real-time field, where
providing temporal guarantees on the response time is a mandatory constraint.
As far as we know, at the time of writing there is still no science literature
applying real-time models to web technologies.
%
This work analyzes in detail all the layers which are part of the WebGL
rendering pipeline and it provides exact measurements for the computation time
of every single rendering task.
This performance assessment phase is accomplished using two ad-hoc tools: the
Web Tracing Framework and Chrome's Event Profiling Tool.
The first is used for the very first investigations of the response time and to
get insights of what is happening from an high level point of view.
On the other hand, the profiler included inside the Google Chrome browser is
able to capture exact computation times of the rendering primitives with microsecond
granularity. Hence, this approach starts from the ``macro'' situation and it
proceeds focusing on very detiled rendering primitives.
%
Capturing very precise timing values is essential for applying real-time
mathematical models which are able to define probabilistic deadlines and to predict
the trend of the entire system.
%
The experimental results show that WebGL is able to achieve good performances
and that its rendering pipeline can be modeled using a Markov Computation Time
Model.
This goal is remarkable especially if we consider that the test application is
developed using a very high-level programming language and that it runs inside
a browser, which is rather singular for real-time programs.

        % chapters
        %!TEX root=../thesis.tex
\chapter{Introduction} \label{cha:intro}

Introduction \cite{*} % TODO: remove me

        %!TEX root=../thesis.tex
\chapter{Real-Time background} \label{cha:rt_background}

Background

        %!TEX root=../thesis.tex
\chapter{Technological stack} \label{cha:tech_stack}

There are a lot of technologies involved in this work. The whole system
architechture is composed of many different layers and the technological
stack under study is quite complex. Hence, this chapter aims at explaining
all the remarkable parts.


\section{FriWalk's architechture}
The whole system architechture can be divided into two main parts. One that
is completely platform agnostic and another one that is platform independent.
This distinction is due to the fact that, since the navigator application heavily
uses 3D graphics primitives, different operating systems handle them in different
ways.
\begin{figure}[!htb]\label{img:system_arch}
    \center{\includegraphics[width=0.6\linewidth]{system_architecture.png}}
    \caption{The FriWalk's system architecture.}
\end{figure}

As it is possible to see from Figure \ref{img:system_arch}, the first part includes
the walker assistant hardware, the browser running the
user interface (Google Chrome has been used in this study) and the component of
the OS which is responsible for implementing the OpenGL~\cite{woo1999opengl}
and the GLSL~\cite{marroquim2009introduction} standards. On the other hand,
the platform dependent part includes the 3D graphics driver (eg. Mesa 3D for
Linux) and the OS kernel (eg. the Linux kernel). In the specific case of the
Linux kernel, it is implements a System Call Interface (SCI) for interacting
with the Kernel Mode Setting (KMS)~\cite{linuxkms} and the Direct Rendering Manager
(DRM)~\cite{paul2000introduction}.


\subsection{Platform agnostic}
The 3D navigator application makes use of three pieces of information to move on
the map: the \emph{latitude}, the \emph{longitude} and the \emph{rotation}.
These three values is the
minimum amount of information needed for the navigator to behave correctly. Thus,
from an high-level point of view, the walker assistant hardware can be regarded
as the portion of the system that is able to provide such data. Mooreover, it can
guarantee that data are available for the navigator at a fixed interval over time
using a the WebSocket protocol~\cite{fette2011websocket}.\\
After receiving the data, the navigator properly moves the placeholder on the map.
This step can involves several operations, such as requesting a new tile for the
map, creating new 3D objects and so on. Consequently, the WebGL engine is responsible
for actually drawing the objects on the screen. How WebGL actually behaves
is described in Section \ref{sec:webgl}.\\
Finally, it is possible to say that the flow of information inside the platform
agnostic part of the system works as follows:
\begin{itemize}
    \item the FriWalk sends through a WebSocket the \emph{latitude}, the
        \emph{longitude} and the \emph{rotation} as a comma-separated string.
    \item the navigator running inside a browser receives the string, extracts
        the three values and uses them to build the objects needed to show the
        change on the screen.
    \item the framework powering the navigator application actually performs the
        WebGL function calls.
    \item the graphical engine implemented inside the browser translates the WebGL
        function calls into generic OpenGL and GLSL commands, which are directly
        feeded to the graphical driver.
\end{itemize}


\subsection{Platform dependent} \label{sec:platform_dependent}
This part of the system architechture highly depends on the operating system
running on the machine under test. In this work, a Linux-based OS has been used.
Therefore the description may not fit, fully or partially, to other operating
systems like Apple Mac OS X and Microsoft Windows.\\
The implementation of the OpenGL standard, the Mesa 3D Graphics Library. 
Graphics device drivers are implemented using two components: a User-Mode Driver
(UMD) and a Kernel-Mode Driver (KMD). The first can be seen as the interface
that can be used by other programs, while the second one directly interacts
with the KMS to set the display settings and with the DRM kernel subsystem
interfacing with the GPU(s). Starting from the 4.2 version of the Linux kernel,
multiple graphics drivers (eg. Mesa and AMD Catalyst) can share the same kernel
mode driver.


\section{Linux 3D graphics stack}
First it has to be underlined that we are referring only to the 3D part of the
graphics stack. This means that the window manager needs to ask the X11 server
only for the handle to a portion of the screen (i.e. the window).
3D graphics primitives do not need to any more interaction with the X11 server. This
is needed to bypass the client-server architecture of X11 which is not suitable for
real-time 3D graphics and rendering, since it would have introduced more layers
of communication, delays and unpredictability. This architecture is called
Direct Rendering Infrastructure (DRI)~\cite{paul2000introduction}; it is
necessary to overcome the issue where only the X server was allowed to access
the graphics hardware. Moreover, DRI is composed by a user-space and a kernel-space
(the DRM) part. This provides the building blocks that allow userspace applications
to directly access the graphics hardware in an efficient and safe way.
In addition, the DRM is a key component for the navigator application to achieve
hardware-accelerated 3D rendering and managing video memory in Mesa.

As briefly described in Section \ref{sec:platform_dependent}, the Linux graphics
stack is quite complex and consists of many different layers in different parts
of the OS. The main ones are: the Mesa 3D Graphics Library, the System Call
Interface, the Kernel Mode Setting and the Direct Memory Manager.\\
As it is possible to see from Figure \ref{img:linux_graphics_stack} DRM and 
display server belongs to the windowing system (eg. X11 or Wayland~\cite{wayland}) 
and it is not strictly necessary for applications directly using OpenGL function
calls.
\begin{figure}[!htb]\label{img:linux_graphics_stack}
    \center{\includegraphics[width=0.75\linewidth]{linux_graphics_stack.png}}
    \caption{Illustration of the Linux graphics stack.}
\end{figure} 

Mesa is a free and open-source implementation of the OpenGL specification allowing
programs to output accelerated 3D graphics in the Linux environment. Fortunately,
DRI is conceived to run without the brokering of the X server, but it is still
needed to allocate to mesa a surface on the display to output to. Moreover, from
an high-level perspective the communication between Mesa and the GPU works by
exchanginc commands (eg. ``draw a point'') and data (eg. ``the coordinates of the
point and its color'') that are copied to a buffer of the graphics hardware.


\section{WebGL} \label{sec:webgl}
WebGL is a Javascript standard API for rendering 3D graphics within any compatible
desktop or mobile web browser\footnote{A compatibility table for all the most
common browsers can be found at the following link: \url{http://caniuse.com/#feat=webgl}.}
without any external program or tool. WebGL programs are composed of two parts
of code: some Javascript code, that can be mixed with any other part of a standard
HTML page and is executed by the browser, and some shader code, which is written 
in the OpenGL Shading Language (GLSL) and is executed by the GPU.
WebGL APIs and are exposed to the Javascript engine of the browser through the
HTML5 canvas element are based on the OpenGL for Embedded Systems (ES) technology,
which is a subset of standard OpenGL library. Moreover, this standard is both
royalty-free and open-source\footnote{The WebGL source code can be found on GitHub
at the following link: \url{https://github.com/KhronosGroup/WebGL}.}.

In the last few years a lot of frameworks and utility libraries have been developed
and the huge capabilities of these 3D graphics APIs become visible to everybody.
A lot of libraries have been built on WebGL to render scenes and 3D objects, such
as BabylonJS~\cite{babylon3d}, three.js~\cite{cabello2010three}, and 
Cesium~\cite{cozzi20113d} (see Section \ref{sec:cesium} for a more detailed description).
Moreover, there also has been a rapid emergence~\cite{parisi2014programming} of
game engines exploiting this technology like the Unreal Engine~\cite{games2007unreal}
and Unity~\cite{engine9unity}.

The reason why the 3D navigator developed in this work uses a framework built on
WebGL is that it is, at the time of writing, the only alternative to build real
3D graphics inside a browser. This choiche allows to write Javascript source
code to develop the application and, at the same time, to exploit the power of
a GPU using a framweork that takes care of implementing the shaders programs.
To be more precise, looking at the whole system architecture shown in Figure
\ref{img:system_arch}, WebGL is a technology that lie inside the browser
(in this specific case Google Chrome). The browser is the one responsible
for the communication with the actual implementation of OpenGL, which is Mesa
in this case study.



\section{Cesium framework} \label{sec:cesium}


\section{Goole Chrome's architechture}

        %!TEX root=../thesis.tex
\chapter{Problem definition} \label{cha:problem_definition}

Problem definition

        %!TEX root=../thesis.tex
\chapter{A Real-Time model for WebGL} \label{cha:rt_model}

A Real-Time model for WebGL

        %!TEX root=../thesis.tex
\chapter{Experiments} \label{cha:experiments}


\begin{table}[!htb]
    \centering
    \caption{WebGL functions-to-groups mapping.}
    \label{tab:webgl_func_mapping}
    \begin{tabular}{|l|l|l|}
        \hline
        \textbf{Initialization} & \textbf{Modify data} & \textbf{Display} \\ \hline
        bindBuffer/bindFramebuffer & uniformMatrix3fv/uniformMatrix4fv & drawElements \\
        enable/disable & uniform[1\(\vert\)2\(\vert\)3\(\vert\)4][f\(\vert\)i\(\vert\)iv\(\vert\)fv] & drawArray \\
        viewport & bindTexture/activeTexture &  \\
        clear/clearColor & bufferSubData &  \\
        cullFace &  &  \\
        depthCompare &  &  \\
        useProgram &  &  \\
        colorMask &  &  \\
        \hline
    \end{tabular}
\end{table}

\begin{figure}[!htb]
    \center{\includegraphics[width=0.6\linewidth]{call_arrival.png}}
    \caption{The function calls arrival approximate model with different groups.}
    \label{img:call_arrival}
\end{figure}

        %!TEX root=../thesis.tex
\chapter{Conclusion and future work} \label{cha:conclusion}

% questo approccio può essere esteso per studiare come si comporta una qualsiasi
% applicatione che gira dentro un browser e NON solamente quella di test

Adopting the Markov Computation Time Model to study the behaviour of a web
application which heavily use WebGL is feasible.
The outcomes presented in Section \ref{sec:mctm_results} show that it is
reasonable to split the original computation time trace into two states, since
the output model is always conservative.
However, the values belonging to each modes are not highly correlated but they
are not independent as they should be in principle.
This may be a problem in the logic of the HMM decoder algorithm. In the specific
case of this work, the classification phase splits the original trace into two
states (Figure \ref{img:mctm_2_states}) whose CDFs are clearly ditinct. On the
other hand, when the system is composed of three modes, two out of three CDFs
are completely overlapped. Hence, two is the number of states that best fit the
MCTM

As future work, it is possible to switch from the ``raw'' computation times taken
into consideration in this thesis to ``per-frame'' values. This can be done
summing the computation times belonging to the same frame and it allows to study
the WebGL rendering performance from a slightly different standpoint.
Furthermore, the work of this thesis is not bound only to the specific
application under test and it can be easily extended to study the performances
of any piece of software running inside a web browser. This flexibility allows
the team working on the ACANTO project to run the entire analysis on the
specific hardware available on the FriWalk. In addition to this, another
challenging improvement is to modify Cesium's rendering pipeline to introduce
the concept of priority queues for the graphics commands. In this way it would
be possible to render the ``most important'' parts of the scene first (eg. near
the user's position) and to delay the ``less important'' ones.

    \endgroup

    % bibliografia in formato bibtex
    % aggiunta del capitolo nell'indice
    \addcontentsline{toc}{chapter}{Bibliography}
    % stile con ordinamento alfabetico in funzione degli autori
    \bibliographystyle{plain}
    \bibliography{biblio}

    % appendix
    %\titleformat{\chapter}
    %    {\normalfont\Huge\bfseries}{Appendix \thechapter}{1em}{}
    % sezione Allegati - opzionale
    %\appendix
    %\input{sections/appendix}
\end{document}

%%%%%%%%%%%%%%%%%%%%%%%%%%%%%%%%%%%%%%%%%%%%%%%%%%%%%%%%%%%%%%%%%%%%%%%%%%
%% Nota
%%%%%%%%%%%%%%%%%%%%%%%%%%%%%%%%%%%%%%%%%%%%%%%%%%%%%%%%%%%%%%%%%%%%%%%%%%
%% Nella bibliografia devono essere riportati tutte le fonti consultate
%% per lo svolgimento della tesi. La bibliografia deve essere redatta
%% in ordine alfabetico sul cognome del primo autore.
%%
%% La forma della citazione bibliografica va inserita secondo la fonte utilizzata:
%%
%% LIBRI
%% Cognome e iniziale del nome autore/autori, la data di edizione, titolo, casa editrice, eventuale numero dell’edizione.
%%
%% ARTICOLI DI RIVISTA
%% Cognome e iniziale del nome autore/autori, titolo articolo, titolo rivista, volume, numero, numero di pagine.
%%
%% ARTICOLI DI CONFERENZA
%% Cognome e iniziale del nome autore/autori (anno), titolo articolo, titolo conferenza, luogo della conferenza (città e paese), date della conferenza, numero di pagine.
%%
%% SITOGRAFIA
%% La sitografia contiene un elenco di indirizzi Web consultati e disposti in ordine alfabetico.
%% E’ necessario:
%%   Copiare la URL (l’indirizzo web) specifica della pagina consultata
%%   Se disponibile, indicare il cognome e nome dell’autore, il titolo ed eventuale sottotitolo del testo
%%   Se disponibile, inserire la data di ultima consultazione della risorsa (gg/mm/aaaa).
%%%%%%%%%%%%%%%%%%%%%%%%%%%%%%%%%%%%%%%%%%%%%%%%%%%%%%%%%%%%%%%%%%%%%%%%%%
